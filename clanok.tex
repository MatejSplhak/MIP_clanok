% Metódy inžinierskej práce
\documentclass[10pt,twoside,slovak,a4paper]{article}
\usepackage[slovak]{babel}
%\usepackage[T1]{fontenc}
\usepackage[IL2]{fontenc} % lepšia sadzba písmena Ľ než v T1
\usepackage[utf8]{inputenc}
\usepackage{graphicx}
\usepackage{url} % príkaz \url na formátovanie URL
\usepackage{hyperref} % odkazy v texte budú aktívne (pri niektorých triedach dokumentov spôsobuje posun textu)
\usepackage{cite}
\graphicspath{ {./images/} }
%\usepackage{times}
\pagestyle{headings}
\title{Modely vizúalnej pozornosti a ich využitie pri formátovaní videa\thanks{Semestrálny projekt v predmete Metódy inžinierskej práce, ak. rok 2021/22, vedenie: Vladimír Mlynarovič}} % meno a priezvisko vyučujúceho na cvičeniach
\author{Matej Šplhák\\[2pt]
	{\small Slovenská technická univerzita v Bratislave}\\
	{\small Fakulta informatiky a informačných technológií}\\
	{\small \texttt{xsplhak@stuba.sk}}
	}
\date{\small 14. December 2021} % upravte
\begin{document}
\maketitle
\begin{abstract}
Tento článok sa zameriava na priblíženie výpočtových modelov vizuálnej pozornosti a ich využitie pri video kompresii. V tejto oblasti existuje viacero navrhovaných modelov, no táto práca sa sústredi na hlavné dva: Automatické modely vizualnej pozornosti, ktoré sa snažia pomocou deep learningu adaptívne kompresovať časti obrazu videa podľa ich predpokladanej významnosti a Semiautomatické modely vizuálnej pozornosti, ktoré  si vyžadujú sledovať pohyby zraku jedného pozorovateľa a postprocessing na báze časovej konzistencie.\\ 
\\
\textbf{Kľúčové slová} - významnosť, mapy významnosti, x264, enceder, kompresia na základe významnosti
\end{abstract}


\section{Úvod}
Podľa Cisco  Visual  Networking  Index  Forecast\cite{cisco:forecast} je predpokladané, že v roku 2022 prehrávanie videí bude predstavovať 82\% celkovej premávky po IP adresách. Táto projekcia poukazuje na dôležitú potrebu vývoja nového spôsobu kompresie videa, pretože aj malé vylepšenie pre rozšírený video encoder by znamenalo významný pokles vo veľkosti premávky videí. Niekoľko nových štandardov pre video encoding je práve v procese vývoja (napr. VVC a AVI). Aj keď adaptacia týchto štandardov sľubuje lepšie výsledky, ich vývoj potrvá ešte niekoľko rokov hlavne z dôvodu, že sa musia zariadenia na sledovanie videí aktualizovať. Iný smer pohľadu na túto problematiku je vývoj nových encoderov, ktoré by boli kompatibilné so širšou škálou štandardov. Zámer tejto práce je ten druhý spôsob, presnejšie, porovnať dva typy encodingu založené na rôznych modeloch vizuálnej pozornosti. Prvý z nich\cite{Czarnecki:Progress}, je encoder založený na plne automatickom modeli vizuálnej pozornosti, ktorý používa mapy významnosti, na rozloženie bitratu na rôzne miesta danej snímky, podľa prepokladanej výnamnosti jednotlivych oblastí v danej snímke. Druhý encoder\cite{Coplien:MPD}, je založený na semiautomatickom modeli vizuálnej pozornosti, ktorý zaznamenáva rôzne body fixácie zraku pozorovateľa na danej snímke. Tieto body fixácie neskôr prejdu postprocessingom, ktorý spresní ich pozíciu. Tento článok je rozdelený na viacero sekcií. V sekcii~\ref{1modely} sa bližšie vysvetlia modely vizuálnej pozornosti a taktiež mapy významnosti. V nasledujúcej sekcii~\ref{porovnanie} je samostatné porovnanie týchto typov encodingu. Výsledky tohto porovnania sú zhrnuté v sekcii~\ref{zaver}.   


\section{Modely vyzuálnej pozornosti} \label{1modely}
Model ľudskej vizuálnej pozornosti je model, ktorý sa snaží napodobniť alebo predpovedať správanie ludskej vizuálnej pozornosti. Modely vizuálnej pozornosti sa rozvetvujú do viacerých smerov, z ktorých jeden sú výpočtové modely vizuálnej pozornosti. Výpočtový model vizuálnej pozornosti je model, ktorý zahrňuje nie len formálny opis toho ako vypočítava pozornosť, ale aj môže byť testovaný tým, že sa mu predložia vstupy obrazov podobné tým, ktoré by boli predložené živému subjektu. Následne sa sleduje ako presný bol predpokladaný výsledok. Na obr.~\ref{visio:diag} je príklad modelu vizuálnej pozornosti. Tieto modely často používajú mapy významnosti.
\begin{figure*}[tbh]
\centering
\includegraphics[scale=0.5]{diagram mvp}
\caption{príklad modelu vizuálnej pozornosti, ktorý napodobňuje reakciu ľudského zraku na vizuálny podnet.}
\label{visio:diag}
\end{figure*}

\subsection{Mapy významnosti} \label{ina:nejake}
Podľa definície Christof Kocha a Shimon Ullmana z roku 1985, mapa významnosti (obr.~\ref{f:rozhod}) je topografická mapa poskladaná z rôznych vizuálnych aspektov obrazu (farba, orientácia, pohyb atď), ktoré sú neskôr zvýraznené podľa ich predpokladanej významnosti. V podstate pozorujú ako veľmi je daný bod obrazu odlišný od jeho okolia. Koch a Ullman taktiež predpokladali, že región s najväčšou významnosťou je vhodným kandidátom pre ľudskú zrakovú pozornosť. Mapy významnosti sa hlavne používajú pri predpovedaní pohybu očí, napr. na identifikovanie najdôležitejšej informácie pri vizuálnych vstupoch pre lepší výkon pri generovaní alebo prenášaní vizuálnych dát.
\begin{figure*}[tbh]
\centering
\includegraphics[scale=0.5]{saliency}
\caption{Príklad mapy významnosti. Napravo je je vizuálny obraz a naľavo je jeho mapa významnosti.}
\label{f:rozhod}
\end{figure*}

\subsection{Bottom-up modely} \label{ina}
Modely, ktoré využívajú tieto mapy významnosti sú takzvané Bottom-up modely. Bottom-up modely predpokladajú, že pozornosť je priťahovaná vlastnosťami obrazu. Autori\cite{bottom} naviac aplikujú postprocessing na báze vzájomnosti pixelov, kde podobné pixely zhlukujú do spoločných skupín. Výskumníci zrakovej psychológie, ako Yarbus [1967], Koch [2000] a Parkhurst [2002], taktiež ukázali, že zrakový systém je vysoko náchylný na rôzne črty obrazu, ako napr. rohy, náhle zmeny farby a prudké pohyby. Autor\cite{saliency:vytvar} opisuje, že výroba modelu tohto bottom-up procesu spočíva v tom, že sa vezme vstupný obraz, ktorý je následne rozdelený do troch pralelných kanálov, jeden pre každú črtu obrazu: farba, intenzita a orientácia. Výsledky z týchto troch kanálov sú rozprestrené po modeli a reorganizované do stred-obkolesujúceho rozmiestnenia. Nakoniec sa aktivita v každom kanáli normalizuje a lineárne aplikuje, čím vytvára mapy významnosti. Problémy nastávajú vtedy, keď pozorovateľove ciele sú dôležitejšie ako významné body obrazu. Týmto smerom sa vydávajú Top-down modely.    

\subsection{Top-down modely} \label{nejaka}
Top-down process je riadený kontrolným procesom, ktorý sústredí pozornosť ľudského pozorovateľa na jeden alebo viacej objektov, ktoré su relevantné k jeho cieľom pri sledovaní scény, ako napríklad hľadať dopravné značenie, snažiť sa nájsť  protivníka vo videohre alebo počítať nejaké špecifické objekty v danej scéne. V tomto prípade objekty, ktoré by normálne priťahovali pozornosť sledovaťeľa, môžu byť kompletne odignorované ak sú irelevantné k cieľu pozorovateľa. V roku 1967 ruský psychológ Yarbus zapísal body fixácie a sakády (sakády sú pohyby našich očí medzi dvoma bodmi fixácie) rôznych pozorovateľov, zatiaľ čo sa pozerali na každodenné objekty a scény. V tomto experimente sa pýtali pozorovateľov na rôzne otázky ohľadom scény znázornenej v Ilia Jefimovič Repinovom  obraze "Nečakali". Výsledkom toho boli významne rozdielne vzory sakád, pričom každá z nich poukazovala na tie objekty v obraze, ktoré obsahovali najviac informácií relevantných k odpovedi na otázku. Tento experiment je zobrazený na obrázku~\ref{yarbus:sakady}.
\begin{figure*}[tbh]
\centering
\includegraphics[scale=0.5]{yarbus}
\caption{ Dopad úloh na pohyb očí. Repinov obraz bol pozorovaný subjektmi s rozlišnými inštrukciami (obrázky idú v poradí zľava doprava zhora dole): 1. Voľné pozorovanie, 2. Odhadnite ich vek, 3. Zistite čo robili pred tým, ako sa ukázal nečakaný hosť, 4. Zapamätajte si oblečenie, ktoré majú na sebe postavy oblečené, 5. Zapamätajte si pozíciu ľudí a objektov v izbe a 6. Predpovedajte, ako dlho bol nečakaný hosť preč od danej rodiny [Yarbus 1967].}
\label{yarbus:sakady}
\end{figure*}

\section{Porovnanie automatického a semiautomatického modelu vizuálnej pozornosti} \label{porovnanie}
V minulosti boli výsledky semiautomatického a automatického modelu neporovnateľné. Semiautomatický model dosahoval oveľa lepšie výsledky vďaka tomu, že používal dáta z eye trackingu ľudského pozorovateľa. Avšak v posledných rokoch vďaka vývoju technológie a hlavne deep learningu, automatický model dobieha výkonom semiautomatický.

\subsection{Semi-automatický model vizuálnej pozornosti} \label{semi-auto}
\begin{figure*}[tbh]
\centering
\includegraphics[scale=0.5]{graf}
\caption{poukazuje na vzťah medzi počtom pozorovateľov a presnosťou máp významnosti. Obrázok prevzatý z \cite{Coplien:MPD}.}
\label{semi:graf}
\end{figure*}
Vo väčšine prípadov, je rozdelenie vizuálnej pozornosti nejednotné, teda aspoň čo sa týka umeleckého obsahu. Toto rozdelenie poukazuje na fakt, že zaznamenávanie pohľadu len niekoľkých pozorovateľov dokáže vygenerovať mapy významnosti, ktoré by sa podobali ideálnemu výsledku. Na obr.~\ref{semi:graf} možno vidieť tento výsledok. Každý bod je skóre podobnosti (bližšie vysvetlené v \cite{Azam2016:sim}) medzi mapou významnosti daného počtu pozorovateľov a mapami významnosti ostatných počtov pozorovateľov. Každý bod bol spriemerovaný z každej snímky a spomedzi piatich skupín pozorovateľov. Zároveň bolo v tom čase pravdou, že väčšina automatických modelov neukazovala lepšie výsledky ako základný, na stred zameraný model. Z čoho vyplýva, že nie sú porovnateľné s modelmi vytvorenými sledovaním pohybu očí a to ani pre jedného pozorovateľa. Vďaka tomuto výsledku navrhli tvorcovia\cite{Coplien:MPD} semiautomatický koncept, ktorý používa body fixácie jedného pozorovateľa spolu s postprocessingom na báze časovej konzistencie pozornosti, t.j. objekty ktoré sú označené ako významné v jednej snímke, sú taktiež predpokladané za významné v susedných snímkach.

\subsubsection{Kompresia v semiautomatickom modeli}
Na kompresiu si tvorcovia\cite{Coplien:MPD} vybrali H.264, pretože to je najrozšírenejší štandard na kompresiu videa a x264 pretože to je najpopulárnejší video encoder. Pri kompresii sú predpovedané mapy významnosti downscalenuté tak, aby pasovali do rozmerov 16x16 makroblokovej mriežky. Ak  \(Q \in R^+\) je mapa hodnôt kvantizačného parametru (QP) makrobloku vybraných encoderom, a \(S : R^2\rightarrow [0;1]\) je downscalnutá mapa významnosti pre danú snímku, tak nové hodnoty QP na makroblokoch môžu byť vypočitané nasledujúcou rovnicou:
\[{Q}' = \max(Q - \psi \cdot (S - ES),0)\]
Týmto spôsobom redukujú hodnoty QP na makroblokoch obasuhujúcich významné regióny alebo naopak zväčšujú hodnoty QP na makroblokoch obsahujúcich nevýznamné regióny. Parameter \(\psi\) je zadaný používateľom a reguluje rozmiestnenie bitratu medzi významnými a nevýznamnými regiónmi: čím väčšia hodnota, tým viac bitov pre významné regióny.\\
Využitím tejto rovnice autori\cite{Coplien:MPD} navrhujú toto reťazové spracovanie:\\
1. Spusti x264 s nasledujúcimi argumentmi\\
--qcomp 0 --pass1 --bitrate \( \phi\) ,\\
kde \( \phi\) je cieľový bitrate\\
2. načítaj hodnoty QP z .mbtree súboru vytvoreného encoderom a uprav ich podľa vyššie uvedenej rovnice \\
3. Spusti x264 s nasledujúcimi argumentmi\\
--qcomp 0 --pass 2 --bitrate \(\phi\)\\
Semiautomatický model dosahoval po objektívnom ohodnotení o 23\% nižší bitrate ako obyčajný x264 encoder, pričom zachoval rovnakú kvalitu.

\subsection{Automatický model vizuálnej pozornosti} \label{auto}
Narozdiel od semiautomatického modelu, automatický model nepoužíva ľudského pozorovateľa. Autori\cite{Czarnecki:Progress} navrhli automatický model, ktorý rozdeľuje bitrate podľa sekvencie významnostných máp každej snímky, ktoré su vkladané ako dodatočný vstup. Na dosiahnutie takejto redistribúcie bitratu bolo potrebné pozmeniť vnútornú štruktúru encoderu: delta kvantizéry makrobloku. Na predpovedanie zrakovej pozornosti využili najmodernejśie modely na predpovedanie významnosti. Autori\cite{Czarnecki:Progress} navrhli model, ktorý sa skladá z dvoch častí: modul na predpovedanie významnosti a modul na encodovanie. Model postupuje nasledovne: Najprv presunie video do modulu na predpovedanie významnosti, následne presunie vytvorené mapy významnosti aj s daným videom do encoding modulu, ktorý nakoniec vyprodukuje encodnutý bitstream. Keďže daný model je automatický, jeho kvalita záleží od kvality modelu na vytváranie máp významnosti. Tvorcovia mali na výber z troch rôznych modelov: OM-CNN, ACL a SAM-ResNet. OM-CNN a ACL sú modely, ktoré boli vytvorené na predpovedanie významnosti vo videách a SAM-ResNet bol vytvorený na predpovedanie významnosti v snímkach. SAM nie je schopný využiť pohybové vlastnosti obrazu, ale zato veľmi silno využíva štrukturálne vlastnosti snímky, pretože bol trénovaný na viac rozmanitej databáze obrazov. Narozdiel od SAM, OM-CNN bol trénovaný na malej skupine videí. Z toho je možné predpokladať, že statické modely vizuálnej pozornosti môžu byť lepšie v zovšeobecňovaní a tým pádom aj môžu vytvárať lepšie mapy významnosti pri videách s nezvyčajným alebo doposiaľ nevideným obsahom. ACL kombinuje oba prístupy; jeho model na predpovedanie významnosti bol trénovaný aj na videách, aj na snímkach. V tabuľke~\ref{tab:mapy} je možné vidieť objektívne ohodnotenie týchto troch modelov spolu aj s tromi základnými modelmi vizuálnej pozornosti.
\begin{table}[h]
\caption{objektívne porovnanie vybraných modelov a troch základných modelov na súbore dát SAVAM použitím troch metrík AUC-Judd (AUC-J), koeficient linearnej korelácie (CC) a metrika podobnosti (SIM). Výsledky zahrňujú modely ktoré prešli fine tuningom (FT) a taktiež ich výsledok po postprocessingu (PP)\\}
\label{tab:mapy}
\begin{tabular}{llll}
Model                  & AUC-J & CC    & SIM   \\
ideálny výsledok       & -     & -     & -     \\
Sam-ResNet + FT + PP   & 0.833 & 0.672 & 0.618 \\
Sam-ResNet + FT        & 0.835 & 0.664 & 0.616 \\
Jeden pozorovateľ + PP & 0.807 & 0.651 & 0.603 \\
ACL + FT + PP          & 0.820 & 0.633 & 0.603 \\
OM-CNN + FT + PP       & 0.790 & 0.557 & 0.563 \\
Sred-zameraný model    & 0.751 & 0.469 & 0.503
\end{tabular}
\end{table}
\\
Ako možno vidieť v tabuľke, verzia SAM-ResNet modelu, ktorá prešla post processingom a fine-tuningom ukazovala lepšie výsledky, ako model na báze jedného pozorovateľa, ktorý je základom pre semiautomatický model. Vďaka výsledkom tohto merania si autori\cite{Czarnecki:Progress} vybrali SAM-ResNet ako model na predpovedanie významnosti obrazu.

\subsubsection{Kompresia v automatickom modeli}
Na kompresiu používajú autori\cite{Czarnecki:Progress} významnosti-vedomú modifikáciu x264 software encoderu, ako ich encoding modul. Táto modifikovaná verzia vyberá delta kvantizéry makrobloku tak, že \(b\) percentá bitratu encodnu \(p\) percent z nevýznamných regiónov podľa vstupu z mapy významnosti. Parametre \(b\) a \(p\) su zadefinované používateľom. Na to aby dosiahli vyššie uvedené rozdelenie bitratu použili nasledovné pravidlo na aktualizovanie:
\begin{figure*}[tbh]
\centering
\includegraphics[scale=0.5]{rovnica}
\label{f:rovnica}
\end{figure*}
\\
kde \(Q\) je originálna a \({Q}'\) je modifikovaná kvantizačná mapa; B\((q)\) predpokladá počet bitov potrebnýh na encodovanie bloku s kvantizérom \(q\) ; \(SP = \max(S - S_{p}, 0)\) ; \(SN = \max(S_{p} - S, 0)\). S je vtupná mapa významnosti a \(S_{p}\) je hodnotou pre jej \(p\)-tý percentil. Automatický model dosahoval po objedktívnom ohodnotení o 25\% nižší bitrate ako obyčajný x264 encoder, pričom zachoval rovnakú kvalitu. 

\section{Reakcie na prednášky}

\subsection{Plagiátorstvo}
Plagiátorstvo je skopírovanie a použitie cudzej práce bez potrebnej citácie alebo referencie. Plagiátorstvo je neetické a často porušuje fair use. Ľudia by si mali dávať veľký pozor pri písaní na to, aby správne citovali a odkazovali na prácu iných. Plagiátorstvo má viacero foriem, Môže sa jednať o obrázok, doslovne preložený text alebo text, ktorý bol parafrázovaný. Často sa to deje z lenivosti človeka na to, aby vyrvoril vlastnú prácu. Občas môže byť plagiátorstvo  neúmyselné. To avšak neospravedlňuje danú osobu, ktorá neskôr môže čeliť vážnym následkom, ako je napr. diskreditácia, strata pozície alebo aj trestné konanie.

\subsection{Sedem statočných inžinierov}
V prezentácií o rôznych priekopníkoch inžinierstva v 20. storoči zaznelo mnoho zaujímavých mien a ešte viac zaujímavých prínosov no jeden z nich je z môjho pohľadu najprelomovejší. Larry Tesler vytvoril snáď najdôležitejší nástroj novodobého študenta: ctrl + c, ctrl + v. Touto jednoduchou metodikou zachránil milióny študentov od potreby kritického rozmýšlania a používania ich znalostí. Nespočetná čiastka projektov a iných školských zadaní je každý deň riešená touto prelomovou metódou. No tento nástroj má aj svoje nevýhody. Bohužiaľ je veľmi jednoducho odhaliteľné, či študent použil túto metódu a tým pádom stojí za jej použitím veľké riziko. Ideálne použitie podľa mňa predstavuje kombinácia metódy navrhnutej Teslerom a miernej mozgovej aktivity.

\subsection{Kreatívne písanie}
Kreatívne písanie je veľmi široký pojem. Čo všetko sa dá predstaviť pod pojmom kreatívne písanie? Je hocijaké písanie, ktoré vyšlo čisto z našej hlavy kreatívne? Podľa môjho názoru áno. Veľká časť ľudí si pod kreatívnym písaním predstaví stereotypickú literatúru ako napr. báseň alebo novela, ale to zahrňuje iba malý zlomok toho, čo reálne kreatívne písanie je. Je to forma sebaexpresie a vyjadrovania kde človek nazrie do podvedomia a vytiahne odtiaľ kúsok seba, ktorý potom pretransformuje do interpretovateľnej formy. Tieto výtvory sú akýmsi oknom do podvedomia pisateľa a dovolia nám nahliadnúť na jeho myšlienkové pochody. Ja osobne rád kreatívne píšem aj keď nie často. Je to aktivita pri ktorej môžem využiť časť podvedomia, ktorú normálne nevyužívam.

\section{Záver} \label{zaver} 
Tento článok bližšie priblížil význam modelov vizuálnej pozornosti a taktiež porovnal dva z najefektivnejších konceptov vizuálnych modelov: semiautomatický a automatický. Semiautomatický model, navrhnutý autormi\cite{Coplien:MPD} používa dáta z eye trackingu jedného pozorovateľa, ktoré následne doladí v postprocessingu. Automatický model, navrhnutý autormi\cite{Czarnecki:Progress}, je zložený z dvoch modulov: encoding modul, ktorý je založený na x264 video encoderi a modulu na predpovedanie významnosti, pri ktorom zohľadňovali tri možné modely na predpovedanie významnosti. Najlepšie výsledky ukazoval model SAM-ResNet, ktorý bol doladený pomocou fine tuningu a postprocessingu. Semiautomatický model dosahoval o 23\% nižší bitrate a automatický dosahoval o 25\% nižží bitrate pričom zachovávali rovnakú kvalitu videa. Z tohto faktu vyplýva, že aj keď v minulosti dosahoval semiautomatický model lepšie výsledky ako automatický, tak pokrok v deep learningu posunul automatický model na podobnú kvalitu. Oba encodery používajú H.264 bitstream čo znamená, že dané modely môžu byť aplikované na video kompresiu bez potreby aktualizácie infraštruktúry alebo používateľských zariadení na to aby podporovali nový encoding štandard.

\bibliography{literatura}
\bibliographystyle{plain} % prípadne alpha, abbrv alebo hociktorý iný
\end{document}
